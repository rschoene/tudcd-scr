\documentclass[
  logo={
    color=blue,
    language=de,
    colormodel=rgb
  },
  fontsize=11tudpt,
  usemargin,
  oneside
]{tudcdreprt}
\usepackage[T1]{fontenc}
\usepackage[utf8]{inputenc}
\usepackage[ngerman=ngerman-x-latest]{hyphsubst}

\usepackage[ngerman]{babel}
\usepackage[babel]{microtype}
\babelprovide[hyphenrules=ngerman-x-latest]{ngerman}

\usepackage{biblatex}
\addbibresource{handbook.bib}
\addbibresource{source.bib}
\usepackage{csquotes}
\usepackage{hvlogos}

\usepackage{siunitx}

\usepackage{listings}
\lstset{
  language={[LaTeX]Tex},
  basicstyle=\small\ttfamily,
  literate=%
  {Ö}{{\"O}}1
  {Ä}{{\"A}}1
  {Ü}{{\"U}}1
  {ß}{{\ss}}1
  {ü}{{\"u}}1
  {ä}{{\"a}}1
  {ö}{{\"o}}1
  {~}{{\textasciitilde}}1,
  numberstyle=\footnotesize,
  backgroundcolor=\SelectTUDCDSecondBackgroundColor
%  numbers=left,
}

\author{Jochen Diepelt}
\title{TUDCD-Script: Offizielle Klassen und Pakete im Stil der TU Dresden}
\subject{Handbuch}
\date{2026.01.30}

\TUDCDsetup{
  titlepage={
    highlight-color=Brilliantblau,
    background-color=Blau2%
  },
  color={
    highlight-color=Brilliantblau,
    second-background-color=Blau2%
  }
}

\begin{document}

\maketitle%

%%%%%%%%%%

\tableofcontents

%%%%%%%%%%

\chapter{Einführung}

Mit der Erneuerung des Corporate Designs der Technischen Universität Dresden wurden auch \LaTeX{} Klassen
mit in Auftrag gegeben.
Dabei wurde sich dazu entschieden, die \enquote{inoffiziellen Klassen} von Falk Hanisch~\cite{ctan-tudscr}
abzulösen und eine neue Paketsammlung zu erstellen.
Aus Archivierungsgründen bleibt jedoch die Paketsammlung \texttt{tudscr} auf \CTAN{} als solche erhalten,
und daher wurde der Name \textbf{\texttt{tudcdscr}} für die neue Paketsammlung gewählt.

\section{Allgemeine Einstellungen}

Die grundlegenden Einstellungen von Optionen erfolgt über das Makro \lstinline|\TUDCDsetup|,
welches die Schlüssel-Wert-Paare der Klassenoptionen entgegennimmt.

\begin{lstlisting}
\TUDCDsetup{
  <Schlüssel>=<Wert>
}
\end{lstlisting}

Die Optionen unterteilen sich dabei in drei Kategorien:
\begin{enumerate}
  \item globale Optionen, welche in \lstinline|\documentclass| geladen werden müssen,
  \item späte Optionen, welche in der Dokumentenpräambel über \lstinline|\TUDCDsetup| geladen werden und
  \item einfache Optionen, welche zu jeder Zeit im Dokument über \lstinline|\TUDCDsetup| geladen werden können.
\end{enumerate}
Einfache und späte Optionen können ebenfalls in \lstinline|\documentclass| eingestellt werden,
während einfache Optionen auch in der Dokumentenpräambel geladen werden können.

\section{Einstellungen des Logos}

Das Logo der TUD hat 3 Merkmale, welche unabhängig voneinander eingestellt werden können,
\begin{itemize}
  \item die Logosprache, in Deutsch und Englisch,
  \item die Logofarbe, in Blau, Weiß und Schwarz, sowie
  \item das Logofarbmodel, in CMYK und in RGB.
\end{itemize}
Diese Merkmale werden über die einfachen Optionen
\begin{lstlisting}
\TUDCDsetup{
  logo/language=<Wert>,
  logo/color=<Wert>,
  logo/colormodel=<Wert>
}
\end{lstlisting}
eingestellt.

\begin{description}
  \item[\texttt{logo/language}] besitzt die zulässigen Werte \texttt{de}, \texttt{en} und \texttt{auto}
  \item[\texttt{logo/color}] besitzt die zulässigen Werte \texttt{white}, \texttt{black} und \texttt{blue}
  \item[\texttt{logo/model}] besitzt die zulässigen Werte \texttt{cmyk}, \texttt{rgb} und \texttt{auto}
\end{description}
Bei der Einstellung \texttt{auto} versucht die Klasse zu einem geeigneten Zeitpunkt die Option aus anderen Informationen
zu setzen, dies muss jedoch zwischen Versionen nicht unbedingt stabil bleiben.
Zu bevorzugen sind definitiv die festen Werte.

Die Einstellungen können ebenfalls über \texttt{color} gesetzt werden
\begin{lstlisting}
\TUDCDsetup{
  logo={
    language=<Wert>,
    color=<Wert>,
    colormodel=<Wert>
  }
}
\end{lstlisting}

Mit den Logoeinstellungen kann anschließend das Logo mittels \lstinline|\TypesetTUDLogo| gesetzt werden.
\begin{lstlisting}
\TypesetTUDLogo{<Dimension>}
\end{lstlisting}

\textbf{ACHTUNG:} Das Logo besitzt eine Unterlänge, weswegen der Shape des Logos immer \emph{kleiner} sein wird,
als die angegeben Größe. Die Dimension muss mit dem Faktor \num{1.13} multipliziert werden, damit der Shape die angegebene Größe hat.

\TUDCDsetup{
  color={
    highlight-color=Tuerkis1
  }
}

\chapter{Dokumente im Stil der TU Dresden}
  Hauptklassen \texttt{tudcdartcl} und \texttt{tudcdreprt}

Zum Zeitpunkt des Schreibens sind die beiden Hauptklassen
\texttt{tudcdartcl} und \texttt{tudcdreprt}.

\section{Einstellungen}




\subsection{Wieder anderer Test}

\subsubsection{Noch ein Test}

\TUDCDsetup{
  color={
    highlight-color=Brilliantblau
  }
}
\printbibliography%

\end{document}
