\documentclass[aspectratio=169,onlytextwidth,handout]{beamer}
\usetheme[cd2025]{tudcd}

\usepackage[utf8]{inputenc}
\usepackage[ngerman]{babel}
\usepackage{amsmath,amssymb,amsthm}
\usepackage{hyperref}
\hypersetup{
    colorlinks,
    allcolors={Gruen1},
    urlcolor={Gruen1}
}
\usepackage{lipsum}

\setbeamercovered{transparent}
\setfrontbackgroundimage{willersbau1}
\setfrontbackgroundimagecaption{Willersbau, Foto: G. Binsack}

\author[V. Nachname]{Vorname Nachname}
\einrichtung{Technische Universität Dresden}
\fachrichtung{Mathematik}
\institute[TUD]{Fakultät Mathematik, TU Dresden}

\title[Kurzer Titel für die Fußzeile]{Hier steht der Titel der Präsentation}
\subtitle{Hier steht ein Untertitel}
\datecity{Ort der Präsentation}
\date{\today}

\newcommand\blankfootnote[1]{%
  \let\thefootnote\relax\footnotetext{\footnotesize{#1}}%
  \let\thefootnote\svthefootnote%
}


\begin{document}
\maketitle


\begin{frame}
  \frametitle{Die ist ein Frametitel}
  \framesubtitle{Hier steht der Subtitel}
  \lipsum[1][1-4]
  \begin{itemize}
    \item Erstes item
    \item Zweites item mit inline Formel: $E = mc^2$
    \item Drittes item mit abgesetzter Formel:
      \[
        \int_a^b f(x)\,dx = F(b) - F(a).
      \]
  \end{itemize}
\end{frame}

\begin{frame}
  \frametitle{Die ist ein zweiter Frame}
  \framesubtitle{Noch ein Untertitel}
  \lipsum[2][1-3]\footnote{Eine Fußnote mit Nummerierung}

  \begin{enumerate}
    \item Erstes enumerated item
    \item Zweites enumerated item
    \item Drittes enumerated item
  \end{enumerate}

  \blankfootnote{Eine Fußnote ohne Nummerierung}
\end{frame}

\begin{frame}{Ein Frame mit Boxen}
  \lipsum[1][5-6]

  \onslide<2->{%
  \begin{columns}[t]
    \begin{column}{.5\textwidth-1em}
    \begin{block}{Blocktitel Links}
      \lipsum[1][7-8]
    \end{block}
    \end{column}\begin{column}{.5\textwidth-1em}
    \begin{block}{Rechts}
      \lipsum[1][9-10]
    \end{block}
    \end{column}
  \end{columns}}
\end{frame}


\begin{frame}
  \frametitle{Frame mit Beispiel und Alertbox}

  In diesem Text gibt es eine \alert{hervorgehobenes} Wort und spezielle Boxen:

  \begin{exampleblock}{Beispiel}
    \lipsum[1][11-12]
  \end{exampleblock}
  \begin{alertblock}{Alertbox}
    \lipsum[1][13-14]
  \end{alertblock}
\end{frame}

\begin{frame}
  \frametitle{Frame mit Theorem und Definition}

  \begin{theorem}[Pythagoras]
    Für ein rechtwinkliges Dreieck mit den Katheten $a$ und $b$ und der Hypotenuse $c$ gilt:
    \[
      a^2 + b^2 = c^2.
    \]
  \end{theorem}

  \begin{definition}[Abgeschlossene Menge]
    Eine Menge $M$ heißt \emph{abgeschlossen}, wenn für alle Folgen $(a_n)_{n\in\mathbb{N}}$ in $M$ mit $a_n\to a$ auch $a\in M$ gilt.
  \end{definition}
\end{frame}
\end{document}
